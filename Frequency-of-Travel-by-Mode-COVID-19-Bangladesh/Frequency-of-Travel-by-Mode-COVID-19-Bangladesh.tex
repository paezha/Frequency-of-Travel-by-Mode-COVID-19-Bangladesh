\documentclass[]{elsarticle} %review=doublespace preprint=single 5p=2 column
%%% Begin My package additions %%%%%%%%%%%%%%%%%%%
\usepackage[hyphens]{url}

  \journal{Transport Findings} % Sets Journal name


\usepackage{lineno} % add
\providecommand{\tightlist}{%
  \setlength{\itemsep}{0pt}\setlength{\parskip}{0pt}}

\usepackage{graphicx}
\usepackage{booktabs} % book-quality tables
%%%%%%%%%%%%%%%% end my additions to header

\usepackage[T1]{fontenc}
\usepackage{lmodern}
\usepackage{amssymb,amsmath}
\usepackage{ifxetex,ifluatex}
\usepackage{fixltx2e} % provides \textsubscript
% use upquote if available, for straight quotes in verbatim environments
\IfFileExists{upquote.sty}{\usepackage{upquote}}{}
\ifnum 0\ifxetex 1\fi\ifluatex 1\fi=0 % if pdftex
  \usepackage[utf8]{inputenc}
\else % if luatex or xelatex
  \usepackage{fontspec}
  \ifxetex
    \usepackage{xltxtra,xunicode}
  \fi
  \defaultfontfeatures{Mapping=tex-text,Scale=MatchLowercase}
  \newcommand{\euro}{€}
\fi
% use microtype if available
\IfFileExists{microtype.sty}{\usepackage{microtype}}{}
\bibliographystyle{elsarticle-harv}
\usepackage{graphicx}
\ifxetex
  \usepackage[setpagesize=false, % page size defined by xetex
              unicode=false, % unicode breaks when used with xetex
              xetex]{hyperref}
\else
  \usepackage[unicode=true]{hyperref}
\fi
\hypersetup{breaklinks=true,
            bookmarks=true,
            pdfauthor={},
            pdftitle={An exploratory analysis of changes in trip-making frequency by mode during the COVID-19 emergency in Bangladesh},
            colorlinks=false,
            urlcolor=blue,
            linkcolor=magenta,
            pdfborder={0 0 0}}
\urlstyle{same}  % don't use monospace font for urls

\setcounter{secnumdepth}{0}
% Pandoc toggle for numbering sections (defaults to be off)
\setcounter{secnumdepth}{0}


% Pandoc header
\usepackage{booktabs}
\usepackage{longtable}
\usepackage{array}
\usepackage{multirow}
\usepackage{wrapfig}
\usepackage{float}
\usepackage{colortbl}
\usepackage{pdflscape}
\usepackage{tabu}
\usepackage{threeparttable}
\usepackage{threeparttablex}
\usepackage[normalem]{ulem}
\usepackage{makecell}



\begin{document}
\begin{frontmatter}

  \title{An exploratory analysis of changes in trip-making frequency by mode
during the COVID-19 emergency in Bangladesh}
    \author[McMaster University]{Shaila Jamal}
   \ead{jamals16@mcmaster.ca} 
    \author[McMaster University]{Antonio Paez\corref{1}}
   \ead{paezha@mcmaster.ca} 
      \address[McMaster University]{School of Earth, Environment and Society, McMaster University, Hamilton,
ON, L8S 4K1, Canada}
      \cortext[1]{Corresponding Author}
  
  \begin{abstract}
  The COVID-19 pandemic has had a profound impact on mobility in every
  country and region around the world. Recent research provides
  illuminates the nature and magnitude of the changes in mobility, but the
  evidence is still scant in developing countries. The objective of this
  paper is to present an exploratory analysis of the changes in the
  frequency of trip-making by mode during the COVID-19 emergency in
  Bangladesh. The results confirm an overall loss of mobility in the form
  of reduced trip-making frequency by all modes, but the changes are not
  uniform across modes.
  \end{abstract}
  
 \end{frontmatter}

\hypertarget{research-questions-and-hypotheses}{%
\section{Research Questions and
Hypotheses}\label{research-questions-and-hypotheses}}

The spread of the COVID-19 pandemic has led to limitations to movement
in many countries and regions, either because of lock-down policies or
self-censoring by segments of the public. The magnitude of changes in
mobility has been studied by recent research, including DeWeese et al.
(2020) and Molloy et al. (2020). While the evidence available indicates
that overall there was a reduction in mobility in much of the world, the
changes were uneven depending on the mode of transportation or the
purpose of the trip (see Lock, 2020; Paez, 2020). Alas, with few
exceptions evidence remains more spotty for developing countries, most
of which have large populations segments that are less able to absorb
losses in mobility (e.g., Astroza et al., 2020; Huynh, 2020; Saha et
al., 2020).

The objective of this paper is to investigate changes in the trip-making
frequency by different modes of transportation during the COVID-19
emergency in Bangladesh. Using data from a recent survey that asked
respondents to report mobility levels before and during the pandemic, we
pose the following questions:

\begin{itemize}
\tightlist
\item
  Was there a reduction of mobility in Bangladesh during COVID-19?
\item
  And if so, what forms of transportation were more affected?
\end{itemize}

This paper is a reproducible research document (see Brunsdon and Comber,
2020); the code and data necessary to replicate the tables and figures
are available in a public repository\footnote{\url{place.holder}}

\hypertarget{methods-and-data}{%
\section{Methods and Data}\label{methods-and-data}}

Data used for this paper come from a survey conducted during
\textbf{DATES} in \textbf{regions/cities} in Bangladesh. \textbf{BRIEFLY
ADD DETAILS ABOUT SURVEY}.

The survey asked respondents to self-report their trip-making frequency
by eight modes of transportation, namely \textbf{car},
\textbf{ridesharing} (e.g., Uber, Pathao), \textbf{rickshaw},
\textbf{cng auto-rickshaw} (a rickshaw-like vehicle powered by
compressed natural gas), \textbf{bus}, motorcycle/scooter (hereafter
just \textbf{motorcycle}), \textbf{walking}, and \textbf{bicycle} (there
was an additional catch-all category \textbf{other} which we ignore
here). Participants in the survey used the following levels to report
their frequency of traveling by each mode both before and during
COVID-19: \emph{Never}, \emph{Rarely}, \emph{Once a week}, \emph{2-3
trips per week}, \emph{4-5 trips per week}, \emph{Almost daily}. There
are \(n=800\) responses in the data set.

To describe changes in the frequency of travel by mode in the transition
to the pandemic, we use well-established exploratory data analysis (EDA)
techniques.

\hypertarget{findings}{%
\section{Findings}\label{findings}}

Figure \ref{fig:column-plot-cases} shows the number of responses (out of
800) in each trip-making frequency class by mode of transportation. The
white bars and gray bars are for travel before and during the pandemic,
respectively. Considering travel before the pandemic, travel by
rickshaw, bus were relatively common for many respondents (few
respondents reported \emph{never} using these modes). The mode most
commonly used on a quotidian basis was walk. In contrast, respondents
reported less frequent travel by car, rideshare services, cng
auto-rickshaw, motorcycle, and bicycle. During the pandemic we see that
while there were reductions in mobility by car, motorcycle, and bicycle
(with more respondents reporting never traveling by these modes), the
changes were relatively minor. The frequency of trip-making by other
modes changed more noticeably: the frequency of travel by rideshare
services, rickshaw, cng auto-rickshaw, and bus collapsed, with vastly
more respondents reporting never using these modes during the pandemic
than before. The frequency of walking trips also decreased (fewer
respondents report walking almost daily), but the reductions in mobility
were not so heavily concentrated at the bottom of the scale.

\begin{figure}
\centering
\includegraphics{Frequency-of-Travel-by-Mode-COVID-19-Bangladesh_files/figure-latex/column-plot-cases-before-after-1.pdf}
\caption{\label{fig:column-plot-cases}Number of responses by trip-making
frequency class and mode, before and during COVID-19}
\end{figure}

Table \ref{tab:cross-tabulation} is a cross-tabulation of the number of
cases in each trip-making frequency class before and during the
pandemic. If no changes had occurred, all values would be concentrated
on the main diagonal of the matrices. Values in the lower triangular
matrix represent a \emph{loss} of mobility (lower travel frequency),
whereas values in the upper triangular matrix are \emph{gains} (higher
travel frequency). The further away a value is from the main diagonal,
the greater the loss or gain.

Despite across-the-board losses of mobility, there appears to have been
some adaptation that varied by mode of transportation. To illustrate,
103 respondents, or 65.61\% of those who traveled by bus almost daily
before, reported never using it during the pandemic. In contrast, only
12 respondents, or 1.5\% of those who never used buses before started
doing so during the pandemic. By way of comparison, 24.14\% of
respondents who cycled almost daily before the pandemic stopped doing so
- but 4.88\% who never cycled before started doing so during the
pandemic.

\begin{table}

\caption{\label{tab:table-transitions}\label{tab:cross-tabulation} Cross-tabulation of counts of travel frequency by mode, before and during COVID-19}
\centering
\resizebox{\linewidth}{!}{
\begin{tabular}[t]{lrrrrrr}
\toprule
  & Never & Rarely & Once a week & 2-3 per week & 4-5 per week & Almost daily\\
\midrule
\addlinespace[0.3em]
\multicolumn{7}{l}{\textbf{car}}\\
\hspace{1em}Never & 267 & 20 & 5 & 1 & 1 & 2\\
\hspace{1em}Rarely & 142 & 99 & 21 & 12 & 9 & 3\\
\hspace{1em}Once a week & 11 & 23 & 19 & 7 & 2 & 3\\
\hspace{1em}2-3 per week & 10 & 23 & 7 & 6 & 8 & 2\\
\hspace{1em}4-5 per week & 4 & 9 & 5 & 6 & 12 & 4\\
\hspace{1em}Almost daily & 6 & 15 & 6 & 10 & 0 & 20\\
\addlinespace[0.3em]
\multicolumn{7}{l}{\textbf{rideshare}}\\
\hspace{1em}Never & 254 & 13 & 6 & 4 & 0 & 0\\
\hspace{1em}Rarely & 192 & 44 & 12 & 1 & 1 & 1\\
\hspace{1em}Once a week & 60 & 20 & 31 & 3 & 1 & 0\\
\hspace{1em}2-3 per week & 49 & 12 & 9 & 7 & 2 & 0\\
\hspace{1em}4-5 per week & 28 & 10 & 3 & 5 & 6 & 2\\
\hspace{1em}Almost daily & 14 & 3 & 4 & 1 & 0 & 2\\
\addlinespace[0.3em]
\multicolumn{7}{l}{\textbf{rickshaw}}\\
\hspace{1em}Never & 51 & 9 & 6 & 0 & 1 & 0\\
\hspace{1em}Rarely & 64 & 70 & 18 & 4 & 3 & 1\\
\hspace{1em}Once a week & 36 & 38 & 45 & 9 & 3 & 3\\
\hspace{1em}2-3 per week & 34 & 36 & 18 & 29 & 6 & 4\\
\hspace{1em}4-5 per week & 21 & 30 & 7 & 16 & 15 & 8\\
\hspace{1em}Almost daily & 45 & 65 & 30 & 18 & 16 & 41\\
\addlinespace[0.3em]
\multicolumn{7}{l}{\textbf{cng auto-rickshaw}}\\
\hspace{1em}Never & 176 & 14 & 3 & 1 & 0 & 0\\
\hspace{1em}Rarely & 213 & 95 & 14 & 4 & 1 & 3\\
\hspace{1em}Once a week & 36 & 31 & 38 & 9 & 2 & 3\\
\hspace{1em}2-3 per week & 29 & 18 & 16 & 14 & 8 & 3\\
\hspace{1em}4-5 per week & 13 & 6 & 5 & 9 & 14 & 1\\
\hspace{1em}Almost daily & 8 & 7 & 2 & 0 & 0 & 4\\
\addlinespace[0.3em]
\multicolumn{7}{l}{\textbf{bus}}\\
\hspace{1em}Never & 116 & 9 & 3 & 0 & 0 & 0\\
\hspace{1em}Rarely & 155 & 46 & 9 & 0 & 0 & 1\\
\hspace{1em}Once a week & 73 & 32 & 22 & 4 & 0 & 0\\
\hspace{1em}2-3 per week & 43 & 19 & 11 & 7 & 5 & 0\\
\hspace{1em}4-5 per week & 60 & 6 & 5 & 4 & 12 & 1\\
\hspace{1em}Almost daily & 103 & 21 & 10 & 11 & 2 & 10\\
\addlinespace[0.3em]
\multicolumn{7}{l}{\textbf{motorcycle}}\\
\hspace{1em}Never & 467 & 22 & 12 & 3 & 1 & 8\\
\hspace{1em}Rarely & 44 & 38 & 13 & 10 & 0 & 4\\
\hspace{1em}Once a week & 11 & 11 & 34 & 3 & 3 & 3\\
\hspace{1em}2-3 per week & 13 & 6 & 6 & 15 & 4 & 2\\
\hspace{1em}4-5 per week & 6 & 4 & 0 & 4 & 14 & 3\\
\hspace{1em}Almost daily & 1 & 3 & 2 & 4 & 3 & 23\\
\addlinespace[0.3em]
\multicolumn{7}{l}{\textbf{walk}}\\
\hspace{1em}Never & 17 & 7 & 3 & 4 & 0 & 2\\
\hspace{1em}Rarely & 26 & 53 & 13 & 6 & 2 & 4\\
\hspace{1em}Once a week & 12 & 19 & 54 & 9 & 4 & 2\\
\hspace{1em}2-3 per week & 10 & 14 & 12 & 26 & 13 & 4\\
\hspace{1em}4-5 per week & 7 & 3 & 7 & 11 & 16 & 8\\
\hspace{1em}Almost daily & 46 & 73 & 40 & 49 & 27 & 197\\
\addlinespace[0.3em]
\multicolumn{7}{l}{\textbf{bicycle}}\\
\hspace{1em}Never & 504 & 22 & 7 & 4 & 2 & 4\\
\hspace{1em}Rarely & 59 & 43 & 20 & 4 & 2 & 4\\
\hspace{1em}Once a week & 14 & 13 & 21 & 3 & 2 & 1\\
\hspace{1em}2-3 per week & 3 & 4 & 5 & 7 & 2 & 0\\
\hspace{1em}4-5 per week & 2 & 2 & 4 & 1 & 9 & 3\\
\hspace{1em}Almost daily & 7 & 1 & 1 & 1 & 3 & 16\\
\bottomrule
\end{tabular}}
\end{table}

To more easily understand the transitions towards different trip-making
frequencies, including possible adaptations by mode, we convert the
cross-tabulations to probability transition matrices, which we then
visualize using circular plots.

Figures \ref{fig:circular-plot-1} to \ref{fig:circular-plot-4} present
these plots. Each of the trip-making frequency sectors on the left
hemisphere of the circle represent 100\% of responses \emph{before} the
pandemic. The size of the links is proportional to the probability
\(p_{ij}\) of transitioning from frequency class \(i\) before to the
right hemisphere are proportional to the transition probabilities to
each frequency \emph{during} the pandemic. There are three transparency
levels for the links: solid colors are for \(p_{ij}>2/3\), intermediate
transparency is for \(1/3 < p_{ij} \le 2/3\), and the more transparent
links are for \(p_{ij}\le 1/3\).

From Figure \ref{fig:circular-plot-1} we see that the probability of
traveling less frequently for those who frequently traveled by car
before is high, but their probability of not using this mode at all
during the pandemic is quite small. The probability of traveling more
frequently by car for those who originally never or rarely used this
mode remained is low. In contrast, we see that the probabilities of
never ridesharing during the pandemic are high irrespective of the
initial level of use of this mode of transportation.

The probabilities of change in trip frequency by rickshaw and and cng
auto-rickshaw are similar (see Figure \ref{fig:circular-plot-2}),
although the probabilities of being less mobile by cng auto-rickshaw are
greater: between 1/3 and 2/3 of respondents who used this mode almost
daily, stopped using it during the pandemic. Very rarely there was an
increase in mobility by these modes.

After ridesharers, bus users (Figure \ref{fig:circular-plot-3}) were the
most likely to stop using it, irrespective of their initial level of use
of this mode, while the probabilities of using it \emph{more} frequently
are extremely small.

\begin{figure}
\centering
\includegraphics{Frequency-of-Travel-by-Mode-COVID-19-Bangladesh_files/figure-latex/circular-plots-transition-probabilities-1-1.pdf}
\caption{\label{fig:circular-plot-1}Transition probabilities in trip
frequency from before to during COVID-19: car and rideshare}
\end{figure}

\begin{figure}
\centering
\includegraphics{Frequency-of-Travel-by-Mode-COVID-19-Bangladesh_files/figure-latex/circular-plots-transition-probabilities-2-1.pdf}
\caption{\label{fig:circular-plot-2}Transition probabilities in trip
frequency from before to during COVID-19: rickshaw and cng
auto-rickshaw}
\end{figure}

\begin{figure}
\centering
\includegraphics{Frequency-of-Travel-by-Mode-COVID-19-Bangladesh_files/figure-latex/circular-plots-transition-probabilities-3-1.pdf}
\caption{\label{fig:circular-plot-3}Transition probabilities in trip
frequency from before to during COVID-19: motorcycle and bus}
\end{figure}

\begin{figure}
\centering
\includegraphics{Frequency-of-Travel-by-Mode-COVID-19-Bangladesh_files/figure-latex/circular-plots-transition-probabilities-4-1.pdf}
\caption{\label{fig:circular-plot-4}Transition probabilities in trip
frequency from before to during COVID-19: bicycle and walk}
\end{figure}

\hypertarget{references}{%
\section*{References}\label{references}}
\addcontentsline{toc}{section}{References}

\hypertarget{refs}{}
\leavevmode\hypertarget{ref-Astroza2020mobility}{}%
Astroza, S., Tirachini, A., Hurtubia, R., Carrasco, J.A., Guevara, A.,
Munizaga, M., Figueroa, M., Torres, V., 2020. Mobility changes,
teleworking, and remote communication during the covid-19 pandemic in
chile. Transport Findings.
doi:\href{https://doi.org/10.32866/001c.13489}{10.32866/001c.13489}

\leavevmode\hypertarget{ref-Brunsdon2020opening}{}%
Brunsdon, C., Comber, A., 2020. Opening practice: Supporting
reproducibility and critical spatial data science. Journal of
Geographical Systems 1--20.

\leavevmode\hypertarget{ref-DeWeese2020tale}{}%
DeWeese, J., Hawa, L., Demyk, H., Davey, Z., Belikow, A., El-geneidy,
A., 2020. A tale of 40 cities: A preliminary analysis of equity impacts
of covid-19 service adjustments across north america. Tansport Findings.
doi:\href{https://doi.org/10.32866/001c.13395}{10.32866/001c.13395}

\leavevmode\hypertarget{ref-Huynh2020culture}{}%
Huynh, T.L.D., 2020. Does culture matter social distancing under the
covid-19 pandemic? Safety Science 130, 7.
doi:\href{https://doi.org/10.1016/j.ssci.2020.104872}{10.1016/j.ssci.2020.104872}

\leavevmode\hypertarget{ref-Lock2020cycling}{}%
Lock, O., 2020. Cycling behaviour changes as a result of covid-19: A
survey of users in sydney, australia. Transport Findings.
doi:\href{https://doi.org/10.32866/001c.13405}{10.32866/001c.13405}

\leavevmode\hypertarget{ref-Molloy2020tracing}{}%
Molloy, J., Tchervenkov, C., Hintermann, B., Axhausen, K.W., 2020.
Tracing the sars-cov-2 impact: The first month in switzerland. Transport
Findings.
doi:\href{https://doi.org/10.32866/001c.12903}{10.32866/001c.12903}

\leavevmode\hypertarget{ref-Paez2020using}{}%
Paez, A., 2020. Using google community mobility reports to investigate
the incidence of covid-19 in the united states. Transport Findings.
doi:\href{https://doi.org/10.32866/001c.12976}{10.32866/001c.12976}

\leavevmode\hypertarget{ref-Saha2020lockdown}{}%
Saha, J., Barman, B., Chouhan, P., 2020. Lockdown for covid-19 and its
impact on community mobility in india: An analysis of the covid-19
community mobility reports, 2020. Children and Youth Services Review
116, 14.
doi:\href{https://doi.org/10.1016/j.childyouth.2020.105160}{10.1016/j.childyouth.2020.105160}


\end{document}


